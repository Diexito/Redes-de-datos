\documentclass[spanish]{udpreport}
\usepackage[utf8]{inputenc}
\usepackage[spanish]{babel}
\usepackage{textcomp}

% Podemos establecer el logo de alguna entidad o dejar el de la UDP (defecto)
%\setlogo{EITFI}

\title{
	Levantamiento de Topología Física y Lógica\\[2ex]
	\normalsize
	Laboratorio N°1, Redes de Datos
    }
\author{John Bidwell Boitano\\ Diego Aguilera Morán \\ Valentín Morales Albornoz \\ Yerson Sarria Quiroz}
\email{john.bidwell@mail.udp.cl\\diego.aguileramo@mail.udp.cl\\valentin.morales@mail.udp.cl\\yerson.sarria@mail.udp.cl}
\date{31 de marzo de 2016}

% Además podemos establecer la facultad y escuela
% los valores por defecto son los siguientes:
%\udpschool{Escuela de Informática y Telecomunicaciones}
%\udpfaculty{Facultad de Ingeniería}
%\udpuniversity{Universidad Diego Portales}

\begin{document}
\maketitle

\tableofcontents

\chapter{Introducción}
Es de vital importancia comprender cómo se compone una comunicación entre componentes electrónicos (Rx - Tx, emisor y receptor respectivamente), así pues es necesario estudiar y tener en cuenta cada situación correspondiente para que se pueda efectuar la intención de enviar un mensaje, la identificación de quien se quiere que reciba el mensaje, la generación del mensaje y un canal por el cual enviar el mensaje, para así establecer un sistema de intercambio de información entre 2 o más dispositivos.
Teniendo en cuenta lo anterior, es que se da lugar al desarrollo del primer laboratorio de Redes de Datos, componiendose de la capacidad de identificar dispositivos, en este caso un switch, un patch pannel y múltiples computadores de escritorio, la forma en que estos se comunican (cable) hasta el hardware presente en los computadores. En la experiencia se analizará la topología presente en el laboratorio, se identificará la red local como una LAN, entre otros, para así dar paso a desarrollar el informe detallado de cómo es que es posible el levantamiento de este tipo de situaciones para que múltiples dispositivos interactúen entre sí y sean capaces de transmitir información los unos a los otros.

\section{¿Por qué necesito una sexy?}

Cada capítulo a su vez se divide en sexy. A diferencia de un artículo cuyo elemento superior es solo una sexy, este documento puede tener capítulos para organizar más información.

\chapter{Identificación de elementos de red}

Como primer paso, es necesario identificar los elementos que componen la red del lugar en el cual se trabajará, en este caso, el Laboratorio de Telemática de la Facultad de Ingeniería de la Universidad Diego Portales.
    Este Laboratorio cuenta con 19 PCs unidos y conectados en red mediante una topología del tipo estrella, tal como lo muestra el siguiente diagrama:

\section{Equipos conectados a la Red}
Miau.

\section{Switchs de la Topologia}
mas miau.

\section{Hardware de Red}
miau miau.

\section{Cableado}
miau.

\section{Patch Panel}
Marca: Siemon

Modelo: 5e HD5 Series (confirmar el modelo, el 5 no estoy seguro)

Puertos:

\chapter{Información de los Dispositivos}

\chapter{Diagrama de Red}
Zangano asqueroso tu diagrama esta malo.

\listoffigures

\end{document}